%%%%%%%%%%%%%%%%%%%%%%%%%%%%%%%%%%%%%%%%
% MyCommands                           %
%%%%%%%%%%%%%%%%%%%%%%%%%%%%%%%%%%%%%%%%

% Abk�rzung
\newcommand{\eg}{e.\,g.\xspace}
\newcommand{\Eg}{E.\,g.\xspace}
\newcommand{\ie}{i.\,e.\xspace}
\newcommand{\Ie}{I.\,e.\xspace}

% Kennzeichnung
\renewcommand{\vec}[1]{\boldsymbol{#1}}
  \newcommand{\mat}[1]{\boldsymbol{#1}}

% Opertatoren und Funktionen
  \newcommand{\abs}[1]{\left| #1 \right|}
  \newcommand{\dif}{\mathrm{d}}
\renewcommand{\Re}{\mathrm{Re}}
\renewcommand{\Im}{\mathrm{Im}}
  \newcommand{\T}{^{\mathrm T}}
  \newcommand{\f}{\mathrm{f}}
  \newcommand{\g}{\mathrm{g}}
  \newcommand{\h}{\mathrm{h}}
  \newcommand{\F}{\mathrm{F}}
  \newcommand{\besK}{\mathrm{K}}
	\newcommand{\ftrans}[2][]{\mathcal F_{#1} \left\{ #2 \right\}}
	\newcommand{\iftrans}[2][]{\mathcal F_{#1}^{-1} \left\{ #2 \right\}}
	\newcommand{\ltrans}[2][]{\mathcal L_{#1} \left\{ #2 \right\}}
	\newcommand{\iltrans}[2][]{\mathcal L_{#1}^{-1} \left\{ #2 \right\}}
  \newcommand{\heaveside}{\Uptheta}

% Konstanten
  \newcommand{\ce}{\mathrm{e}}
  \newcommand{\ci}{\mathrm{i}}
  \newcommand{\cpi}{\uppi}

% Physikalisch Symbole
	\newcommand{\camline}{z_c}
	\newcommand{\camleng}{c}
	\newcommand{\clh}{\frac{\camleng}{2}}
	\newcommand{\clhs}{\frac{\camleng^2}{4}}
	\newcommand{\bcond}{w}
  \newcommand{\Ma}{\mathit{Ma}}
	\newcommand{\uinf}{U_\infty}
	\newcommand{\theo}{\mathrm{C}}
	\newcommand{\modtheo}{\hat{\theo}}
	\newcommand{\wagn}{\Upphi}
	\newcommand{\modwagn}{\hat \wagn}
	\newcommand{\rLa}{\hat{s}}
	\newcommand{\rt}{\hat{t}}
	\newcommand{\dort}[1]{\stackrel{\scriptscriptstyle \circ}{#1}}
	\newcommand{\ddort}[1]{\stackrel{\scriptscriptstyle \circ\circ}{#1}}
	\newcommand{\SC}[3]{c_{#1}^{#2#3}} 
	\newcommand{\SCfacos}{A}
	\newcommand{\SCfsqrt}{B}
	\newcommand{\SCfTa}{T_1(x)}
	\newcommand{\SCfTb}{T_2(x)}
	\newcommand{\SCfTc}{T_3(x)}
	\newcommand{\ph}{{\left(\protect\cdot\right)}}

% F�r Integral Herleitungen
\newtheoremstyle{corollary}% hnamei
{}% hSpace abovei
{}% hSpace belowi
{}% hBody fonti
{}% hIndent amounti1
{\bfseries}% hTheorem head fonti
{\\}% hPunctuation after theorem headi
{.5em}% hSpace after theorem headi2
{}% hTheorem head spec (can be left empty, meaning `normal'
\theoremstyle{corollary} 
\newtheorem{corollary}{Corollary}[]
\newcommand{\corref}[1]{corollary \ref{#1}}

% Units



% Sonstiges
%\newcommand{\todo}[1]{\marginline{\textcolor{red}{Todo: #1}}}


%eof